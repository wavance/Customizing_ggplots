% Options for packages loaded elsewhere
\PassOptionsToPackage{unicode}{hyperref}
\PassOptionsToPackage{hyphens}{url}
\PassOptionsToPackage{dvipsnames,svgnames,x11names}{xcolor}
%
\documentclass[
  letterpaper,
  DIV=11,
  numbers=noendperiod]{scrartcl}

\usepackage{amsmath,amssymb}
\usepackage{iftex}
\ifPDFTeX
  \usepackage[T1]{fontenc}
  \usepackage[utf8]{inputenc}
  \usepackage{textcomp} % provide euro and other symbols
\else % if luatex or xetex
  \usepackage{unicode-math}
  \defaultfontfeatures{Scale=MatchLowercase}
  \defaultfontfeatures[\rmfamily]{Ligatures=TeX,Scale=1}
\fi
\usepackage{lmodern}
\ifPDFTeX\else  
    % xetex/luatex font selection
\fi
% Use upquote if available, for straight quotes in verbatim environments
\IfFileExists{upquote.sty}{\usepackage{upquote}}{}
\IfFileExists{microtype.sty}{% use microtype if available
  \usepackage[]{microtype}
  \UseMicrotypeSet[protrusion]{basicmath} % disable protrusion for tt fonts
}{}
\makeatletter
\@ifundefined{KOMAClassName}{% if non-KOMA class
  \IfFileExists{parskip.sty}{%
    \usepackage{parskip}
  }{% else
    \setlength{\parindent}{0pt}
    \setlength{\parskip}{6pt plus 2pt minus 1pt}}
}{% if KOMA class
  \KOMAoptions{parskip=half}}
\makeatother
\usepackage{xcolor}
\setlength{\emergencystretch}{3em} % prevent overfull lines
\setcounter{secnumdepth}{-\maxdimen} % remove section numbering
% Make \paragraph and \subparagraph free-standing
\makeatletter
\ifx\paragraph\undefined\else
  \let\oldparagraph\paragraph
  \renewcommand{\paragraph}{
    \@ifstar
      \xxxParagraphStar
      \xxxParagraphNoStar
  }
  \newcommand{\xxxParagraphStar}[1]{\oldparagraph*{#1}\mbox{}}
  \newcommand{\xxxParagraphNoStar}[1]{\oldparagraph{#1}\mbox{}}
\fi
\ifx\subparagraph\undefined\else
  \let\oldsubparagraph\subparagraph
  \renewcommand{\subparagraph}{
    \@ifstar
      \xxxSubParagraphStar
      \xxxSubParagraphNoStar
  }
  \newcommand{\xxxSubParagraphStar}[1]{\oldsubparagraph*{#1}\mbox{}}
  \newcommand{\xxxSubParagraphNoStar}[1]{\oldsubparagraph{#1}\mbox{}}
\fi
\makeatother

\usepackage{color}
\usepackage{fancyvrb}
\newcommand{\VerbBar}{|}
\newcommand{\VERB}{\Verb[commandchars=\\\{\}]}
\DefineVerbatimEnvironment{Highlighting}{Verbatim}{commandchars=\\\{\}}
% Add ',fontsize=\small' for more characters per line
\usepackage{framed}
\definecolor{shadecolor}{RGB}{241,243,245}
\newenvironment{Shaded}{\begin{snugshade}}{\end{snugshade}}
\newcommand{\AlertTok}[1]{\textcolor[rgb]{0.68,0.00,0.00}{#1}}
\newcommand{\AnnotationTok}[1]{\textcolor[rgb]{0.37,0.37,0.37}{#1}}
\newcommand{\AttributeTok}[1]{\textcolor[rgb]{0.40,0.45,0.13}{#1}}
\newcommand{\BaseNTok}[1]{\textcolor[rgb]{0.68,0.00,0.00}{#1}}
\newcommand{\BuiltInTok}[1]{\textcolor[rgb]{0.00,0.23,0.31}{#1}}
\newcommand{\CharTok}[1]{\textcolor[rgb]{0.13,0.47,0.30}{#1}}
\newcommand{\CommentTok}[1]{\textcolor[rgb]{0.37,0.37,0.37}{#1}}
\newcommand{\CommentVarTok}[1]{\textcolor[rgb]{0.37,0.37,0.37}{\textit{#1}}}
\newcommand{\ConstantTok}[1]{\textcolor[rgb]{0.56,0.35,0.01}{#1}}
\newcommand{\ControlFlowTok}[1]{\textcolor[rgb]{0.00,0.23,0.31}{\textbf{#1}}}
\newcommand{\DataTypeTok}[1]{\textcolor[rgb]{0.68,0.00,0.00}{#1}}
\newcommand{\DecValTok}[1]{\textcolor[rgb]{0.68,0.00,0.00}{#1}}
\newcommand{\DocumentationTok}[1]{\textcolor[rgb]{0.37,0.37,0.37}{\textit{#1}}}
\newcommand{\ErrorTok}[1]{\textcolor[rgb]{0.68,0.00,0.00}{#1}}
\newcommand{\ExtensionTok}[1]{\textcolor[rgb]{0.00,0.23,0.31}{#1}}
\newcommand{\FloatTok}[1]{\textcolor[rgb]{0.68,0.00,0.00}{#1}}
\newcommand{\FunctionTok}[1]{\textcolor[rgb]{0.28,0.35,0.67}{#1}}
\newcommand{\ImportTok}[1]{\textcolor[rgb]{0.00,0.46,0.62}{#1}}
\newcommand{\InformationTok}[1]{\textcolor[rgb]{0.37,0.37,0.37}{#1}}
\newcommand{\KeywordTok}[1]{\textcolor[rgb]{0.00,0.23,0.31}{\textbf{#1}}}
\newcommand{\NormalTok}[1]{\textcolor[rgb]{0.00,0.23,0.31}{#1}}
\newcommand{\OperatorTok}[1]{\textcolor[rgb]{0.37,0.37,0.37}{#1}}
\newcommand{\OtherTok}[1]{\textcolor[rgb]{0.00,0.23,0.31}{#1}}
\newcommand{\PreprocessorTok}[1]{\textcolor[rgb]{0.68,0.00,0.00}{#1}}
\newcommand{\RegionMarkerTok}[1]{\textcolor[rgb]{0.00,0.23,0.31}{#1}}
\newcommand{\SpecialCharTok}[1]{\textcolor[rgb]{0.37,0.37,0.37}{#1}}
\newcommand{\SpecialStringTok}[1]{\textcolor[rgb]{0.13,0.47,0.30}{#1}}
\newcommand{\StringTok}[1]{\textcolor[rgb]{0.13,0.47,0.30}{#1}}
\newcommand{\VariableTok}[1]{\textcolor[rgb]{0.07,0.07,0.07}{#1}}
\newcommand{\VerbatimStringTok}[1]{\textcolor[rgb]{0.13,0.47,0.30}{#1}}
\newcommand{\WarningTok}[1]{\textcolor[rgb]{0.37,0.37,0.37}{\textit{#1}}}

\providecommand{\tightlist}{%
  \setlength{\itemsep}{0pt}\setlength{\parskip}{0pt}}\usepackage{longtable,booktabs,array}
\usepackage{calc} % for calculating minipage widths
% Correct order of tables after \paragraph or \subparagraph
\usepackage{etoolbox}
\makeatletter
\patchcmd\longtable{\par}{\if@noskipsec\mbox{}\fi\par}{}{}
\makeatother
% Allow footnotes in longtable head/foot
\IfFileExists{footnotehyper.sty}{\usepackage{footnotehyper}}{\usepackage{footnote}}
\makesavenoteenv{longtable}
\usepackage{graphicx}
\makeatletter
\def\maxwidth{\ifdim\Gin@nat@width>\linewidth\linewidth\else\Gin@nat@width\fi}
\def\maxheight{\ifdim\Gin@nat@height>\textheight\textheight\else\Gin@nat@height\fi}
\makeatother
% Scale images if necessary, so that they will not overflow the page
% margins by default, and it is still possible to overwrite the defaults
% using explicit options in \includegraphics[width, height, ...]{}
\setkeys{Gin}{width=\maxwidth,height=\maxheight,keepaspectratio}
% Set default figure placement to htbp
\makeatletter
\def\fps@figure{htbp}
\makeatother

\KOMAoption{captions}{tableheading}
\makeatletter
\@ifpackageloaded{caption}{}{\usepackage{caption}}
\AtBeginDocument{%
\ifdefined\contentsname
  \renewcommand*\contentsname{Table of contents}
\else
  \newcommand\contentsname{Table of contents}
\fi
\ifdefined\listfigurename
  \renewcommand*\listfigurename{List of Figures}
\else
  \newcommand\listfigurename{List of Figures}
\fi
\ifdefined\listtablename
  \renewcommand*\listtablename{List of Tables}
\else
  \newcommand\listtablename{List of Tables}
\fi
\ifdefined\figurename
  \renewcommand*\figurename{Figure}
\else
  \newcommand\figurename{Figure}
\fi
\ifdefined\tablename
  \renewcommand*\tablename{Table}
\else
  \newcommand\tablename{Table}
\fi
}
\@ifpackageloaded{float}{}{\usepackage{float}}
\floatstyle{ruled}
\@ifundefined{c@chapter}{\newfloat{codelisting}{h}{lop}}{\newfloat{codelisting}{h}{lop}[chapter]}
\floatname{codelisting}{Listing}
\newcommand*\listoflistings{\listof{codelisting}{List of Listings}}
\makeatother
\makeatletter
\makeatother
\makeatletter
\@ifpackageloaded{caption}{}{\usepackage{caption}}
\@ifpackageloaded{subcaption}{}{\usepackage{subcaption}}
\makeatother

\ifLuaTeX
  \usepackage{selnolig}  % disable illegal ligatures
\fi
\usepackage{bookmark}

\IfFileExists{xurl.sty}{\usepackage{xurl}}{} % add URL line breaks if available
\urlstyle{same} % disable monospaced font for URLs
\hypersetup{
  pdftitle={Activity 4: Data visualization 2: Customizing plots},
  pdfauthor={Walt and Luis},
  colorlinks=true,
  linkcolor={blue},
  filecolor={Maroon},
  citecolor={Blue},
  urlcolor={Blue},
  pdfcreator={LaTeX via pandoc}}


\title{Activity 4: Data visualization 2: Customizing plots}
\usepackage{etoolbox}
\makeatletter
\providecommand{\subtitle}[1]{% add subtitle to \maketitle
  \apptocmd{\@title}{\par {\large #1 \par}}{}{}
}
\makeatother
\subtitle{Jan.~14th, 2026, Calvin Munson}
\author{Walt and Luis}
\date{}

\begin{document}
\maketitle


\section{Overview}\label{overview}

Welcome! Today's activity is all about the basics of customizing plots.
We will continue using \texttt{ggplot2} and will use two built-in
datasets for this activity (no reading in data is necessary today!).

You will submit two outputs for this activity:

\begin{enumerate}
\def\labelenumi{\arabic{enumi}.}
\tightlist
\item
  A \textbf{PDF} of a rendered Quarto document with all of your R code
  (see below). Please include all of the code that appears in this
  document, in addition to adding your own code in the ``Q\#'' sections.
\end{enumerate}

\emph{If you have trouble submitting as a PDF, please ask Calvin or
Malin for help. If we still can't solve it, you can submit the .qmd file
instead.}

\begin{enumerate}
\def\labelenumi{\arabic{enumi}.}
\setcounter{enumi}{1}
\tightlist
\item
  A plot and research question from the final section
\end{enumerate}

A note about submitting code:

As a general practice, it is good to \textbf{use headers to denote which
section of the homework} you are in and \textbf{add in your own comments
about what each line is actually doing} - this is excellent practice for
organizing your own big R scripts in the future. A common way that you
will code for the first few years of your coding career will consist of
you revisiting old R scripts and copying/pasting code and/or adapting
that code to your current problems. I promise you, your future self will
be thrilled with your current self if you take the extra few minutes to
1) organize your code in a nice way and 2) provide detailed comments
about what each line is doing (as well as what questions you have about
it).

Also, don't be afraid to use lots of blank lines! It is MUCH nicer for
your future self to look at code that has a few blank lines in between
sections of code than a bajillion lines in a row of straight code.

\emph{For the remaining activities, please put headers (either as
comments in the code or as Quarto headers) to denote which question you
are answering (e.g.~``Q1:'') and precede your code with comments that
describe what that code does.}

\section{1) Penguins!}\label{penguins}

\begin{figure}[H]

{\centering \includegraphics{gentoo.jpg}

}

\caption{Gentoo penguin}

\end{figure}%

\subsection{Setting up}\label{setting-up}

First, let's load our packages. You should already have the
\texttt{here} and \texttt{ggplot2} packages installed; read them in from
the library in your code. Also install and read in the
\texttt{palmerpenguins} package, which contains built-in data - you may
recognize this from Malin's ggplot lecture!

\#installing packages and librarying them

\begin{Shaded}
\begin{Highlighting}[]
\CommentTok{\#install.packages("palmerpenguins")}
\FunctionTok{library}\NormalTok{(ggplot2)}
\FunctionTok{library}\NormalTok{(here)}
\FunctionTok{library}\NormalTok{(palmerpenguins)}
\end{Highlighting}
\end{Shaded}

We just installed a package which has data built into it. We are going
to go ahead and store the \texttt{penguins} data frame as a data object
anyway (not technically necessary because it's already built-in, but
this reinforces storing dataframes).

\begin{Shaded}
\begin{Highlighting}[]
\CommentTok{\# Read in the palmer penguins data}
\NormalTok{penguins }\OtherTok{\textless{}{-}}\NormalTok{ penguins}
\end{Highlighting}
\end{Shaded}

Next, let's navigate to the help page of this dataset. Doing so provides
information on the dataset and its various columns. We can learn that
this dataset ``includes measurements for penguin species, island in
Palmer Archipelago, size (flipper length, body mass, bill dimensions),
and sex''. Read on to learn about each column in the dataset.

\begin{Shaded}
\begin{Highlighting}[]
\CommentTok{\# Putting a "?" in front of a function (or built{-}in dataset) will automatically bring up its help page!}
\NormalTok{?penguins}
\end{Highlighting}
\end{Shaded}

\subsection{Start with a question}\label{start-with-a-question}

Let's start by forming a research question. Let's ask: \emph{How are
bill length and bill depth associated, and how does this vary by species
of penguin?}

Let's create a basic graph to begin. Plotting \texttt{bill\_length\_mm}
on the x-axis and \texttt{bill\_depth\_mm} on the y-axis.

\begin{Shaded}
\begin{Highlighting}[]
\FunctionTok{ggplot}\NormalTok{(}\AttributeTok{data =}\NormalTok{ penguins, }\FunctionTok{aes}\NormalTok{(}\AttributeTok{x =}\NormalTok{ bill\_length\_mm, }\AttributeTok{y =}\NormalTok{ bill\_depth\_mm)) }\SpecialCharTok{+}
  \FunctionTok{geom\_point}\NormalTok{()}
\end{Highlighting}
\end{Shaded}

\includegraphics{README_files/figure-pdf/unnamed-chunk-3-1.pdf}

\subsubsection{Q1.1: Do you see a
relationship?}\label{q1.1-do-you-see-a-relationship}

Does it look like there is a relationship between these variables?
Describe the relationship.

\section{There might be multiple linear relationships that may be lost
in the noise, co-varites may be sex or
species.}\label{there-might-be-multiple-linear-relationships-that-may-be-lost-in-the-noise-co-varites-may-be-sex-or-species.}

\begin{center}\rule{0.5\linewidth}{0.5pt}\end{center}

\subsubsection{Q1.2: Color the points by species of
penguin}\label{q1.2-color-the-points-by-species-of-penguin}

That previous graph only addresses part of our question: we asked not
only ``how are bill length and bill depth associated'', but also
``\textbf{how does this vary by species of penguin}?''

To address the second part of our question:

\begin{enumerate}
\def\labelenumi{\arabic{enumi}.}
\item
  Change the aesthetics of this plot to color the points by the species
  of penguin and
\item
  Interpret the graph by describing the relationship you see and if this
  new visualization changes your answer to Q1.1. \# created new graph
  with the new ``color = species'' command, to highlight clusters based
  on species
\end{enumerate}

\begin{Shaded}
\begin{Highlighting}[]
\FunctionTok{ggplot}\NormalTok{(}\AttributeTok{data =}\NormalTok{ penguins, }\FunctionTok{aes}\NormalTok{(}\AttributeTok{x =}\NormalTok{ bill\_length\_mm, }\AttributeTok{y =}\NormalTok{ bill\_depth\_mm)) }\SpecialCharTok{+}
  \FunctionTok{geom\_point}\NormalTok{(}\FunctionTok{aes}\NormalTok{(}\AttributeTok{color =}\NormalTok{ species))}
\end{Highlighting}
\end{Shaded}

\includegraphics{README_files/figure-pdf/unnamed-chunk-4-1.pdf}

The take-home message here is to be thorough with how you look at your
data! If you ignore certain variables, you might miss important patterns
(this is also crucial when deciding what statistical models we should
run on our data, but more on that in a few weeks!).

Ok, now let's jump into some fun data visualization and learn about how
to customize our plots!

\begin{center}\rule{0.5\linewidth}{0.5pt}\end{center}

\subsection{Scales for aesthetics}\label{scales-for-aesthetics}

We can manually modify the appearance of the aesthetics of our ggplot
graph (e.g.~the \texttt{color}, \texttt{shape}, opacity (aka
\texttt{alpha}), \texttt{x}, \texttt{y}, etc) by using a series of
functions called \texttt{scale\_*\_**()} where the first * is replaced
by the aesthetic we are modifying (e.g.~color), and the second
translates to the kind of prepacked or manual set of values that we want
to use. Often it will be \texttt{scale\_*\_manual()} if we want to
manually chose the values instead of using premade ones.

Monday's Seaside Chat provided some practice with modifying the scales
of the color aesthetics using \texttt{scale\_color\_manual()} and
\texttt{scale\_color\_brewer()}.

\subsubsection{Q1.3: Add your favorite colors to your
graph}\label{q1.3-add-your-favorite-colors-to-your-graph}

Using your new color skills, modify the \texttt{scale\_color\_*} code
below to include a new color selection!

\begin{Shaded}
\begin{Highlighting}[]
\FunctionTok{ggplot}\NormalTok{(}\AttributeTok{data =}\NormalTok{ penguins, }\FunctionTok{aes}\NormalTok{(}\AttributeTok{x =}\NormalTok{ bill\_length\_mm, }\AttributeTok{y =}\NormalTok{ bill\_depth\_mm)) }\SpecialCharTok{+}
  \FunctionTok{geom\_point}\NormalTok{(}\FunctionTok{aes}\NormalTok{(}\AttributeTok{color =}\NormalTok{ species)) }\SpecialCharTok{+}
  \FunctionTok{scale\_color\_manual}\NormalTok{(}\AttributeTok{values =} \FunctionTok{c}\NormalTok{(}\StringTok{"green"}\NormalTok{,}\StringTok{"red"}\NormalTok{,}\StringTok{"blue"}\NormalTok{),) }\CommentTok{\#adding manual colors to each species, colors are assigned by aphabetical order}
\end{Highlighting}
\end{Shaded}

\includegraphics{README_files/figure-pdf/unnamed-chunk-5-1.pdf}

Let's make the shape of the points vary by species to even more clearly
separate the points by species:

\emph{Sidenote: When you're typing within a function, you can hit
``Enter'' (or ``Return'' for Mac users) after a comma to continue code
on a new line - see how I made a new line within}
\texttt{geom\_point(aes())}. \emph{This can make things look much neater
if you are typing out many arguments at once! It also does not affect
how the code is run at all.}

\begin{Shaded}
\begin{Highlighting}[]
\FunctionTok{ggplot}\NormalTok{(}\AttributeTok{data =}\NormalTok{ penguins, }\FunctionTok{aes}\NormalTok{(}\AttributeTok{x =}\NormalTok{ bill\_length\_mm, }\AttributeTok{y =}\NormalTok{ bill\_depth\_mm)) }\SpecialCharTok{+}
  \FunctionTok{geom\_point}\NormalTok{(}\FunctionTok{aes}\NormalTok{(}\AttributeTok{color =}\NormalTok{ species,}
                 \AttributeTok{shape =}\NormalTok{ species)) }\SpecialCharTok{+}
  \FunctionTok{scale\_color\_brewer}\NormalTok{(}\AttributeTok{palette =} \StringTok{"Dark2"}\NormalTok{)}
\end{Highlighting}
\end{Shaded}

\includegraphics{README_files/figure-pdf/unnamed-chunk-6-1.pdf}

If we want to manually change the shapes that ggplot supplies, we use
\texttt{scale\_shape\_manual()}:

\begin{Shaded}
\begin{Highlighting}[]
\FunctionTok{ggplot}\NormalTok{(}\AttributeTok{data =}\NormalTok{ penguins, }\FunctionTok{aes}\NormalTok{(}\AttributeTok{x =}\NormalTok{ bill\_length\_mm, }\AttributeTok{y =}\NormalTok{ bill\_depth\_mm)) }\SpecialCharTok{+}
  \FunctionTok{geom\_point}\NormalTok{(}\FunctionTok{aes}\NormalTok{(}\AttributeTok{color =}\NormalTok{ species,}
                 \AttributeTok{shape =}\NormalTok{ species)) }\SpecialCharTok{+}
  \FunctionTok{scale\_color\_brewer}\NormalTok{(}\AttributeTok{palette =} \StringTok{"Dark2"}\NormalTok{) }\SpecialCharTok{+}
  \FunctionTok{scale\_shape\_manual}\NormalTok{(}\AttributeTok{values =} \FunctionTok{c}\NormalTok{(}\DecValTok{8}\NormalTok{, }\DecValTok{11}\NormalTok{, }\DecValTok{14}\NormalTok{))}
\end{Highlighting}
\end{Shaded}

\includegraphics{README_files/figure-pdf/unnamed-chunk-7-1.pdf}

I personally like using shapes that have a black outline and are filled
with the color, rather than being solid throughout. Change the manually
selected shapes to 21, 22, and 24

\begin{Shaded}
\begin{Highlighting}[]
\FunctionTok{ggplot}\NormalTok{(}\AttributeTok{data =}\NormalTok{ penguins, }\FunctionTok{aes}\NormalTok{(}\AttributeTok{x =}\NormalTok{ bill\_length\_mm, }\AttributeTok{y =}\NormalTok{ bill\_depth\_mm)) }\SpecialCharTok{+}
  \FunctionTok{geom\_point}\NormalTok{(}\FunctionTok{aes}\NormalTok{(}\AttributeTok{color =}\NormalTok{ species,}
                 \AttributeTok{shape =}\NormalTok{ species)) }\SpecialCharTok{+}
  \FunctionTok{scale\_color\_brewer}\NormalTok{(}\AttributeTok{palette =} \StringTok{"Dark2"}\NormalTok{) }\SpecialCharTok{+}
  \FunctionTok{scale\_shape\_manual}\NormalTok{(}\AttributeTok{values =} \FunctionTok{c}\NormalTok{(}\DecValTok{21}\NormalTok{, }\DecValTok{22}\NormalTok{, }\DecValTok{24}\NormalTok{))}
\end{Highlighting}
\end{Shaded}

\includegraphics{README_files/figure-pdf/unnamed-chunk-8-1.pdf}

Why aren't the shapes filled with the color we supplied? Some shapes and
geoms in R have both \texttt{color} and \texttt{fill} capabilities. When
they have both, then anything you supply to \texttt{color} will color
the outline of the shape, while anything supplied to \texttt{fill} will
fill in the shape (as we want here).

To make this switch, we are going to change the
\texttt{color\ =\ species} aesthetic to \texttt{fill\ =\ species} and
instead of \texttt{scale\_color\_brewer()} we will use
\texttt{scale\_fill\_brewer()}

\begin{Shaded}
\begin{Highlighting}[]
\FunctionTok{ggplot}\NormalTok{(}\AttributeTok{data =}\NormalTok{ penguins, }\FunctionTok{aes}\NormalTok{(}\AttributeTok{x =}\NormalTok{ bill\_length\_mm, }\AttributeTok{y =}\NormalTok{ bill\_depth\_mm)) }\SpecialCharTok{+}
  \FunctionTok{geom\_point}\NormalTok{(}\FunctionTok{aes}\NormalTok{(}\AttributeTok{fill =}\NormalTok{ species,}
                 \AttributeTok{shape =}\NormalTok{ species)) }\SpecialCharTok{+}
  \FunctionTok{scale\_fill\_brewer}\NormalTok{(}\AttributeTok{palette =} \StringTok{"Dark2"}\NormalTok{) }\SpecialCharTok{+}
  \FunctionTok{scale\_shape\_manual}\NormalTok{(}\AttributeTok{values =} \FunctionTok{c}\NormalTok{(}\DecValTok{21}\NormalTok{, }\DecValTok{22}\NormalTok{, }\DecValTok{24}\NormalTok{))}
\end{Highlighting}
\end{Shaded}

\includegraphics{README_files/figure-pdf/unnamed-chunk-9-1.pdf}

At this point, I'd also like to make the size of the points a little
bigger, but I don't want it to vary with something in the data. As in
the last activity, to make an element of the geom vary statically across
all of the data, put it \emph{outside} of the \texttt{aes()} but still
within the relevant geom:

\begin{Shaded}
\begin{Highlighting}[]
\FunctionTok{ggplot}\NormalTok{(}\AttributeTok{data =}\NormalTok{ penguins, }\FunctionTok{aes}\NormalTok{(}\AttributeTok{x =}\NormalTok{ bill\_length\_mm, }\AttributeTok{y =}\NormalTok{ bill\_depth\_mm)) }\SpecialCharTok{+}
  \FunctionTok{geom\_point}\NormalTok{(}\FunctionTok{aes}\NormalTok{(}\AttributeTok{fill =}\NormalTok{ species,}
                 \AttributeTok{shape =}\NormalTok{ species),}
             \AttributeTok{size =} \FloatTok{2.5}\NormalTok{) }\SpecialCharTok{+}
  \FunctionTok{scale\_fill\_brewer}\NormalTok{(}\AttributeTok{palette =} \StringTok{"Dark2"}\NormalTok{) }\SpecialCharTok{+}
  \FunctionTok{scale\_shape\_manual}\NormalTok{(}\AttributeTok{values =} \FunctionTok{c}\NormalTok{(}\DecValTok{21}\NormalTok{, }\DecValTok{22}\NormalTok{, }\DecValTok{24}\NormalTok{))}
\end{Highlighting}
\end{Shaded}

\includegraphics{README_files/figure-pdf/unnamed-chunk-10-1.pdf}

Let's put a \texttt{geom\_smooth()} with \texttt{method\ =\ "lm"} in
there to look at the relationships more closely!

\begin{Shaded}
\begin{Highlighting}[]
\FunctionTok{ggplot}\NormalTok{(}\AttributeTok{data =}\NormalTok{ penguins, }\FunctionTok{aes}\NormalTok{(}\AttributeTok{x =}\NormalTok{ bill\_length\_mm, }\AttributeTok{y =}\NormalTok{ bill\_depth\_mm)) }\SpecialCharTok{+}
  \FunctionTok{geom\_point}\NormalTok{(}\FunctionTok{aes}\NormalTok{(}\AttributeTok{fill =}\NormalTok{ species,}
                 \AttributeTok{shape =}\NormalTok{ species),}
             \AttributeTok{size =} \FloatTok{2.5}\NormalTok{) }\SpecialCharTok{+}
  \FunctionTok{geom\_smooth}\NormalTok{(}\AttributeTok{method =} \StringTok{"lm"}\NormalTok{) }\SpecialCharTok{+}
  \FunctionTok{scale\_fill\_brewer}\NormalTok{(}\AttributeTok{palette =} \StringTok{"Dark2"}\NormalTok{) }\SpecialCharTok{+}
  \FunctionTok{scale\_shape\_manual}\NormalTok{(}\AttributeTok{values =} \FunctionTok{c}\NormalTok{(}\DecValTok{21}\NormalTok{, }\DecValTok{22}\NormalTok{, }\DecValTok{24}\NormalTok{))}
\end{Highlighting}
\end{Shaded}

\includegraphics{README_files/figure-pdf/unnamed-chunk-11-1.pdf}

Uh oh, what's going on here? Geoms adopt the aesthetics that are
supplied in \texttt{ggplot(aes(*))} as default. All we provided
\texttt{ggplot()} was an \texttt{x\ =} and a \texttt{y\ =}, so as far as
\texttt{geom\_smooth()} is concerned, the \texttt{species} column
doesn't exist. Let's add in \texttt{aes(color\ =\ species)} to the
\texttt{geom\_smooth()} function to tell it to color the smoothed line
by species.

\begin{Shaded}
\begin{Highlighting}[]
\FunctionTok{ggplot}\NormalTok{(}\AttributeTok{data =}\NormalTok{ penguins, }\FunctionTok{aes}\NormalTok{(}\AttributeTok{x =}\NormalTok{ bill\_length\_mm, }\AttributeTok{y =}\NormalTok{ bill\_depth\_mm)) }\SpecialCharTok{+}
  \FunctionTok{geom\_point}\NormalTok{(}\FunctionTok{aes}\NormalTok{(}\AttributeTok{fill =}\NormalTok{ species,}
                 \AttributeTok{shape =}\NormalTok{ species),}
             \AttributeTok{size =} \FloatTok{2.5}\NormalTok{) }\SpecialCharTok{+}
  \FunctionTok{geom\_smooth}\NormalTok{(}\AttributeTok{method =} \StringTok{"lm"}\NormalTok{,}
              \FunctionTok{aes}\NormalTok{(}\AttributeTok{color =}\NormalTok{ species)) }\SpecialCharTok{+}
  \FunctionTok{scale\_fill\_brewer}\NormalTok{(}\AttributeTok{palette =} \StringTok{"Dark2"}\NormalTok{) }\SpecialCharTok{+}
  \FunctionTok{scale\_shape\_manual}\NormalTok{(}\AttributeTok{values =} \FunctionTok{c}\NormalTok{(}\DecValTok{21}\NormalTok{, }\DecValTok{22}\NormalTok{, }\DecValTok{24}\NormalTok{))}
\end{Highlighting}
\end{Shaded}

\includegraphics{README_files/figure-pdf/unnamed-chunk-12-1.pdf}

Great! Now there are three separate \texttt{geom\_smooth()} lines - one
for each species. However, our nice custom color scheme isn't applied.
That's because \texttt{geom\_smooth()} uses a \texttt{color} aesthetic,
and earlier we changed \texttt{scale\_color\_brewer()} into
\texttt{scale\_fill\_brewer()} to accommodate Calvin's unreasonable
desire to have shapes that fill in with color. Let's re-add
\texttt{scale\_color\_brewer()} underneath
\texttt{scale\_fill\_brewer()}

\begin{Shaded}
\begin{Highlighting}[]
\FunctionTok{ggplot}\NormalTok{(}\AttributeTok{data =}\NormalTok{ penguins, }\FunctionTok{aes}\NormalTok{(}\AttributeTok{x =}\NormalTok{ bill\_length\_mm, }\AttributeTok{y =}\NormalTok{ bill\_depth\_mm)) }\SpecialCharTok{+}
  \FunctionTok{geom\_point}\NormalTok{(}\FunctionTok{aes}\NormalTok{(}\AttributeTok{fill =}\NormalTok{ species,}
                 \AttributeTok{shape =}\NormalTok{ species),}
             \AttributeTok{size =} \FloatTok{2.5}\NormalTok{) }\SpecialCharTok{+}
  \FunctionTok{geom\_smooth}\NormalTok{(}\AttributeTok{method =} \StringTok{"lm"}\NormalTok{,}
              \FunctionTok{aes}\NormalTok{(}\AttributeTok{color =}\NormalTok{ species)) }\SpecialCharTok{+}
  \FunctionTok{scale\_fill\_brewer}\NormalTok{(}\AttributeTok{palette =} \StringTok{"Dark2"}\NormalTok{) }\SpecialCharTok{+}
  \FunctionTok{scale\_color\_brewer}\NormalTok{(}\AttributeTok{palette =} \StringTok{"Dark2"}\NormalTok{) }\SpecialCharTok{+}
  \FunctionTok{scale\_shape\_manual}\NormalTok{(}\AttributeTok{values =} \FunctionTok{c}\NormalTok{(}\DecValTok{21}\NormalTok{, }\DecValTok{22}\NormalTok{, }\DecValTok{24}\NormalTok{))}
\end{Highlighting}
\end{Shaded}

\includegraphics{README_files/figure-pdf/unnamed-chunk-13-1.pdf}

\subsection{Scales for axis aesthetics
specifically}\label{scales-for-axis-aesthetics-specifically}

The x- and y- axes are part of the aesthetics too, just like
\texttt{color}, \texttt{fill}, and \texttt{shape.} You can modify the
scales of the axes too. Two common ways people change up the scales of
their graphs are:

\begin{enumerate}
\def\labelenumi{\arabic{enumi}.}
\tightlist
\item
  Changing the limits of their plots and
\item
  Transforming axis values, for example to log scales.
\end{enumerate}

Let's change the limits of the y-axis to go between 0 and 30mm. It's a
continuous variable, so we use \texttt{scale\_y\_continuous()} and
provide a vector of a minimum and a maximum to set the limits.

\begin{Shaded}
\begin{Highlighting}[]
\FunctionTok{ggplot}\NormalTok{(}\AttributeTok{data =}\NormalTok{ penguins, }\FunctionTok{aes}\NormalTok{(}\AttributeTok{x =}\NormalTok{ bill\_length\_mm, }\AttributeTok{y =}\NormalTok{ bill\_depth\_mm)) }\SpecialCharTok{+}
  \FunctionTok{geom\_point}\NormalTok{(}\FunctionTok{aes}\NormalTok{(}\AttributeTok{fill =}\NormalTok{ species,}
                 \AttributeTok{shape =}\NormalTok{ species),}
             \AttributeTok{size =} \FloatTok{2.5}\NormalTok{) }\SpecialCharTok{+}
  \FunctionTok{geom\_smooth}\NormalTok{(}\AttributeTok{method =} \StringTok{"lm"}\NormalTok{,}
              \FunctionTok{aes}\NormalTok{(}\AttributeTok{color =}\NormalTok{ species)) }\SpecialCharTok{+}
  \FunctionTok{scale\_fill\_brewer}\NormalTok{(}\AttributeTok{palette =} \StringTok{"Dark2"}\NormalTok{) }\SpecialCharTok{+}
  \FunctionTok{scale\_color\_brewer}\NormalTok{(}\AttributeTok{palette =} \StringTok{"Dark2"}\NormalTok{) }\SpecialCharTok{+}
  \FunctionTok{scale\_shape\_manual}\NormalTok{(}\AttributeTok{values =} \FunctionTok{c}\NormalTok{(}\DecValTok{21}\NormalTok{, }\DecValTok{22}\NormalTok{, }\DecValTok{24}\NormalTok{)) }\SpecialCharTok{+}
  \FunctionTok{scale\_y\_continuous}\NormalTok{(}\AttributeTok{limits =} \FunctionTok{c}\NormalTok{(}\DecValTok{0}\NormalTok{, }\DecValTok{30}\NormalTok{))}
\end{Highlighting}
\end{Shaded}

\includegraphics{README_files/figure-pdf/unnamed-chunk-14-1.pdf}

\subsubsection{Q1.4: How to use the existing maximum for
limits}\label{q1.4-how-to-use-the-existing-maximum-for-limits}

Let's say we want our limits to include 0, so we manually specify 0 as
the minimum, but we want to use ggplot's default for the maximum y
limit. Go to the built-in help page of \texttt{scale\_y\_continuous()}
and figure out what to replace the * with in
\texttt{limits\ =\ c(0,\ *)}. Then modify that code below.

\begin{Shaded}
\begin{Highlighting}[]
\NormalTok{?}\FunctionTok{scale\_y\_continuous}\NormalTok{()}

\FunctionTok{ggplot}\NormalTok{(}\AttributeTok{data =}\NormalTok{ penguins, }\FunctionTok{aes}\NormalTok{(}\AttributeTok{x =}\NormalTok{ bill\_length\_mm, }\AttributeTok{y =}\NormalTok{ bill\_depth\_mm)) }\SpecialCharTok{+}
  \FunctionTok{geom\_point}\NormalTok{(}\FunctionTok{aes}\NormalTok{(}\AttributeTok{fill =}\NormalTok{ species,}
                 \AttributeTok{shape =}\NormalTok{ species),}
             \AttributeTok{size =} \FloatTok{2.5}\NormalTok{) }\SpecialCharTok{+}
  \FunctionTok{geom\_smooth}\NormalTok{(}\AttributeTok{method =} \StringTok{"lm"}\NormalTok{,}
              \FunctionTok{aes}\NormalTok{(}\AttributeTok{color =}\NormalTok{ species)) }\SpecialCharTok{+}
  \FunctionTok{scale\_fill\_brewer}\NormalTok{(}\AttributeTok{palette =} \StringTok{"Dark2"}\NormalTok{) }\SpecialCharTok{+}
  \FunctionTok{scale\_color\_brewer}\NormalTok{(}\AttributeTok{palette =} \StringTok{"Dark2"}\NormalTok{) }\SpecialCharTok{+}
  \FunctionTok{scale\_shape\_manual}\NormalTok{(}\AttributeTok{values =} \FunctionTok{c}\NormalTok{(}\DecValTok{21}\NormalTok{, }\DecValTok{22}\NormalTok{, }\DecValTok{24}\NormalTok{)) }\SpecialCharTok{+}
  \FunctionTok{scale\_y\_continuous}\NormalTok{(}\AttributeTok{limits =} \FunctionTok{c}\NormalTok{(}\DecValTok{0}\NormalTok{, }\ConstantTok{NA}\NormalTok{)) }\CommentTok{\#replacing the "30" value with "NA" to set the max y{-}value to ggplots default max value}
\end{Highlighting}
\end{Shaded}

\includegraphics{README_files/figure-pdf/unnamed-chunk-15-1.pdf}

\begin{center}\rule{0.5\linewidth}{0.5pt}\end{center}

Another reason to modify the way your axes look is if we want to plot on
a log scale. This can be very useful if your data spans many orders of
magnitude (multiples of 10), which visually makes the very small values
get masked by the large values.

We do this with \texttt{scale\_y\_log10()}, though this isn't a great
dataset to use as an example because the data doesn't span a huge range
- it won't look very different!

\begin{Shaded}
\begin{Highlighting}[]
\FunctionTok{ggplot}\NormalTok{(}\AttributeTok{data =}\NormalTok{ penguins, }\FunctionTok{aes}\NormalTok{(}\AttributeTok{x =}\NormalTok{ bill\_length\_mm, }\AttributeTok{y =}\NormalTok{ bill\_depth\_mm)) }\SpecialCharTok{+}
  \FunctionTok{geom\_point}\NormalTok{(}\FunctionTok{aes}\NormalTok{(}\AttributeTok{fill =}\NormalTok{ species,}
                 \AttributeTok{shape =}\NormalTok{ species),}
             \AttributeTok{size =} \FloatTok{2.5}\NormalTok{) }\SpecialCharTok{+}
  \FunctionTok{geom\_smooth}\NormalTok{(}\AttributeTok{method =} \StringTok{"lm"}\NormalTok{,}
              \FunctionTok{aes}\NormalTok{(}\AttributeTok{color =}\NormalTok{ species)) }\SpecialCharTok{+}
  \FunctionTok{scale\_fill\_brewer}\NormalTok{(}\AttributeTok{palette =} \StringTok{"Dark2"}\NormalTok{) }\SpecialCharTok{+}
  \FunctionTok{scale\_color\_brewer}\NormalTok{(}\AttributeTok{palette =} \StringTok{"Dark2"}\NormalTok{) }\SpecialCharTok{+}
  \FunctionTok{scale\_shape\_manual}\NormalTok{(}\AttributeTok{values =} \FunctionTok{c}\NormalTok{(}\DecValTok{21}\NormalTok{, }\DecValTok{22}\NormalTok{, }\DecValTok{24}\NormalTok{)) }\SpecialCharTok{+}
  \FunctionTok{scale\_y\_log10}\NormalTok{()}
\end{Highlighting}
\end{Shaded}

\includegraphics{README_files/figure-pdf/unnamed-chunk-16-1.pdf}

\subsection{Axis and legend titles}\label{axis-and-legend-titles}

Let's backtrack a little bit to the graph before we started messing with
the y-axis and talk about how to make better axis and legend titles. By
default, ggplot creates these titles based on the column names that you
specify within \texttt{aes()}. However, ``bill\_length\_mm'' is a pretty
lame axis title for a presentation.

To change this, we use the \texttt{labs()} function, which allows us to
rename whatever \texttt{aes()} elements we have specified. We do this
with the format \texttt{x\ =\ "New\ x\ axis\ title"} and/or
\texttt{fill\ =\ "New\ legend\ title\ for\ the\ fill"}. Let's change the
x and y axis titles:

\begin{Shaded}
\begin{Highlighting}[]
\FunctionTok{ggplot}\NormalTok{(}\AttributeTok{data =}\NormalTok{ penguins, }\FunctionTok{aes}\NormalTok{(}\AttributeTok{x =}\NormalTok{ bill\_length\_mm, }\AttributeTok{y =}\NormalTok{ bill\_depth\_mm)) }\SpecialCharTok{+}
  \FunctionTok{geom\_point}\NormalTok{(}\FunctionTok{aes}\NormalTok{(}\AttributeTok{fill =}\NormalTok{ species,}
                 \AttributeTok{shape =}\NormalTok{ species,),}
             \AttributeTok{size =} \FloatTok{2.5}\NormalTok{) }\SpecialCharTok{+}
  \FunctionTok{geom\_smooth}\NormalTok{(}\AttributeTok{method =} \StringTok{"lm"}\NormalTok{,}
              \FunctionTok{aes}\NormalTok{(}\AttributeTok{color =}\NormalTok{ species)) }\SpecialCharTok{+}
  \FunctionTok{scale\_fill\_brewer}\NormalTok{(}\AttributeTok{palette =} \StringTok{"Dark2"}\NormalTok{) }\SpecialCharTok{+}
  \FunctionTok{scale\_color\_brewer}\NormalTok{(}\AttributeTok{palette =} \StringTok{"Dark2"}\NormalTok{) }\SpecialCharTok{+}
  \FunctionTok{scale\_shape\_manual}\NormalTok{(}\AttributeTok{values =} \FunctionTok{c}\NormalTok{(}\DecValTok{21}\NormalTok{, }\DecValTok{22}\NormalTok{, }\DecValTok{24}\NormalTok{)) }\SpecialCharTok{+}
  \FunctionTok{labs}\NormalTok{(}\AttributeTok{x =} \StringTok{"Bill length (mm)"}\NormalTok{,}
       \AttributeTok{y =} \StringTok{"Bill depth (mm)"}\NormalTok{,}
       \CommentTok{\# title = "Bill Length by Bill Depth",}
       \AttributeTok{color =} \StringTok{"Species"}\NormalTok{,}
       \AttributeTok{fill =} \StringTok{"Species"}\NormalTok{,}
       \AttributeTok{shape =} \StringTok{"Species"}\NormalTok{)}
\end{Highlighting}
\end{Shaded}

\includegraphics{README_files/figure-pdf/unnamed-chunk-17-1.pdf}

\subsubsection{Q1.5: Change the legend
title}\label{q1.5-change-the-legend-title}

The legend title is currently not capitalized - add to the
\texttt{labs()} function to change the title of the correct aesthetic(s)
to make the legend title read ``Species''. (You may encounter an
unexpected result - try and figure out what you can do to make it appear
how you want! Hint: how many different aesthetics is \texttt{species}
currently mapped to?).

\subsection{Themes}\label{themes}

This is looking decent, but there's still more we can do. ggplot
provides us with LOTS of flexibility in how the non-data-related
components of the graph look (think the font size, background color,
axis grid lines, etc.). To modify this, we add \textbf{themes} using the
\texttt{theme()} function.

Go to the help page for \texttt{theme()}, read the description, and
scroll down to see the extent of the components you can modify. Also see
this link for a handy visual guide:
\url{https://statsandr.com/blog/best-rstudio-addins-in-rstudio-or-how-to-make-your-coding-life-easier_files/ggplot_theme_system_cheatsheet.pdf}

\begin{Shaded}
\begin{Highlighting}[]
\NormalTok{?}\FunctionTok{theme}\NormalTok{()}
\end{Highlighting}
\end{Shaded}

Within the \texttt{theme()} function, we specify the element that we
want to modify. For instance, the \texttt{axis.title.x\ =} modifies the
x axis title. Because the x axis title is a \texttt{text} element, we
then add \texttt{axis.title.x\ =\ element\_text()} and specify things
within the \texttt{element\_text()} function. Here I will change the
text \texttt{color} to red and increase the \texttt{size}.

\begin{Shaded}
\begin{Highlighting}[]
\FunctionTok{ggplot}\NormalTok{(}\AttributeTok{data =}\NormalTok{ penguins, }\FunctionTok{aes}\NormalTok{(}\AttributeTok{x =}\NormalTok{ bill\_length\_mm, }\AttributeTok{y =}\NormalTok{ bill\_depth\_mm)) }\SpecialCharTok{+}
  \FunctionTok{geom\_point}\NormalTok{(}\FunctionTok{aes}\NormalTok{(}\AttributeTok{fill =}\NormalTok{ species,}
                 \AttributeTok{shape =}\NormalTok{ species),}
             \AttributeTok{size =} \FloatTok{2.5}\NormalTok{) }\SpecialCharTok{+}
  \FunctionTok{geom\_smooth}\NormalTok{(}\AttributeTok{method =} \StringTok{"lm"}\NormalTok{,}
              \FunctionTok{aes}\NormalTok{(}\AttributeTok{color =}\NormalTok{ species)) }\SpecialCharTok{+}
  \FunctionTok{scale\_fill\_brewer}\NormalTok{(}\AttributeTok{palette =} \StringTok{"Dark2"}\NormalTok{) }\SpecialCharTok{+}
  \FunctionTok{scale\_color\_brewer}\NormalTok{(}\AttributeTok{palette =} \StringTok{"Dark2"}\NormalTok{) }\SpecialCharTok{+}
  \FunctionTok{scale\_shape\_manual}\NormalTok{(}\AttributeTok{values =} \FunctionTok{c}\NormalTok{(}\DecValTok{21}\NormalTok{, }\DecValTok{22}\NormalTok{, }\DecValTok{24}\NormalTok{)) }\SpecialCharTok{+}
  \FunctionTok{labs}\NormalTok{(}\AttributeTok{x =} \StringTok{"Bill length (mm)"}\NormalTok{,}
       \AttributeTok{y =} \StringTok{"Bill depth (mm)"}\NormalTok{, }
       \AttributeTok{fill =} \StringTok{"Species"}\NormalTok{,}
       \AttributeTok{color =} \StringTok{"Species"}\NormalTok{,}
       \AttributeTok{shape =} \StringTok{"Species"}\NormalTok{) }\SpecialCharTok{+}
  \FunctionTok{theme}\NormalTok{(}\AttributeTok{axis.title.x =} \FunctionTok{element\_text}\NormalTok{(}\AttributeTok{color =} \StringTok{"red"}\NormalTok{,}
                                    \AttributeTok{size =} \DecValTok{16}\NormalTok{))}
\end{Highlighting}
\end{Shaded}

\includegraphics{README_files/figure-pdf/unnamed-chunk-19-1.pdf}

Depending on what the element is, you use a different
\texttt{element\_*()} function:

\begin{itemize}
\item
  \texttt{element\_text()}: Controls text appearance (font, size, color,
  etc.) for titles, labels, and axis text.
\item
  \texttt{element\_rect()}: Controls the appearance of rectangular
  elements like the plot background and legend background (fill, border
  color, etc.).
\item
  \texttt{element\_line()}: Controls the appearance of line elements
  like grid lines and axis lines (color, size, linetype, etc.).
\item
  \texttt{element\_blank()}: Hides or removes an element entirely (e.g.,
  \texttt{panel.grid.major\ =\ element\_blank()}).
\end{itemize}

\subsubsection{Q1.6: What font faces are
available?}\label{q1.6-what-font-faces-are-available}

Go to the \texttt{element\_text()} help page: under ``Usage'' you will
see what arguments each element function can modify, and if you scroll
down to ``Arguments,'' you can read more about the specifics of each
element you can modify.

What are the four options for the ``face'' of the text?

\begin{Shaded}
\begin{Highlighting}[]
\NormalTok{?element\_text}
\CommentTok{\#font face ("plain", "italic", "bold", "bold.italic")}
\end{Highlighting}
\end{Shaded}

\begin{center}\rule{0.5\linewidth}{0.5pt}\end{center}

Modifying ALL THAT from scratch can be overwhelming! Thankfully, R has a
handful of built-in, pre-curated themes that you can choose from. Add on
a new line to this graph and start typing \texttt{theme\_} - note how
multiple options pop up! Let's click on and add
\texttt{theme\_classic()}.

\begin{Shaded}
\begin{Highlighting}[]
\FunctionTok{ggplot}\NormalTok{(}\AttributeTok{data =}\NormalTok{ penguins, }\FunctionTok{aes}\NormalTok{(}\AttributeTok{x =}\NormalTok{ bill\_length\_mm, }\AttributeTok{y =}\NormalTok{ bill\_depth\_mm)) }\SpecialCharTok{+}
  \FunctionTok{geom\_point}\NormalTok{(}\FunctionTok{aes}\NormalTok{(}\AttributeTok{fill =}\NormalTok{ species,}
                 \AttributeTok{shape =}\NormalTok{ species),}
             \AttributeTok{size =} \FloatTok{2.5}\NormalTok{) }\SpecialCharTok{+}
  \FunctionTok{geom\_smooth}\NormalTok{(}\AttributeTok{method =} \StringTok{"lm"}\NormalTok{,}
              \FunctionTok{aes}\NormalTok{(}\AttributeTok{color =}\NormalTok{ species)) }\SpecialCharTok{+}
  \FunctionTok{scale\_fill\_brewer}\NormalTok{(}\AttributeTok{palette =} \StringTok{"Dark2"}\NormalTok{) }\SpecialCharTok{+}
  \FunctionTok{scale\_color\_brewer}\NormalTok{(}\AttributeTok{palette =} \StringTok{"Dark2"}\NormalTok{) }\SpecialCharTok{+}
  \FunctionTok{scale\_shape\_manual}\NormalTok{(}\AttributeTok{values =} \FunctionTok{c}\NormalTok{(}\DecValTok{21}\NormalTok{, }\DecValTok{22}\NormalTok{, }\DecValTok{24}\NormalTok{)) }\SpecialCharTok{+}
  \FunctionTok{labs}\NormalTok{(}\AttributeTok{x =} \StringTok{"Bill length (mm)"}\NormalTok{,}
       \AttributeTok{y =} \StringTok{"Bill depth (mm)"}\NormalTok{, }
       \AttributeTok{fill =} \StringTok{"Species"}\NormalTok{,}
       \AttributeTok{color =} \StringTok{"Species"}\NormalTok{,}
       \AttributeTok{shape =} \StringTok{"Species"}\NormalTok{) }\SpecialCharTok{+}
  \FunctionTok{theme}\NormalTok{(}\AttributeTok{axis.title.x =} \FunctionTok{element\_text}\NormalTok{(}\AttributeTok{color =} \StringTok{"red"}\NormalTok{,}
                                    \AttributeTok{size =} \DecValTok{16}\NormalTok{)) }\SpecialCharTok{+}
  \FunctionTok{theme\_classic}\NormalTok{()}
\end{Highlighting}
\end{Shaded}

\includegraphics{README_files/figure-pdf/unnamed-chunk-21-1.pdf}

That looks quite nice! The weird gray default background is gone, and
there's now a traditional x and y axis with no axis panel grids.

Note that the large, red x-axis title has been overridden by the new
theme we put in. What happens when you switch the order and put
\texttt{theme\_classic()} before \texttt{theme(...)}? (don't forget to
remove/add \texttt{+}'s where applicable\ldots).

Try out some of the other built-in themes and see what you like!

\begin{Shaded}
\begin{Highlighting}[]
\FunctionTok{ggplot}\NormalTok{(}\AttributeTok{data =}\NormalTok{ penguins, }\FunctionTok{aes}\NormalTok{(}\AttributeTok{x =}\NormalTok{ bill\_length\_mm, }\AttributeTok{y =}\NormalTok{ bill\_depth\_mm)) }\SpecialCharTok{+}
  \FunctionTok{geom\_point}\NormalTok{(}\FunctionTok{aes}\NormalTok{(}\AttributeTok{fill =}\NormalTok{ species,}
                 \AttributeTok{shape =}\NormalTok{ species),}
             \AttributeTok{size =} \FloatTok{2.5}\NormalTok{) }\SpecialCharTok{+}
  \FunctionTok{geom\_smooth}\NormalTok{(}\AttributeTok{method =} \StringTok{"lm"}\NormalTok{,}
              \FunctionTok{aes}\NormalTok{(}\AttributeTok{color =}\NormalTok{ species)) }\SpecialCharTok{+}
  \FunctionTok{scale\_fill\_brewer}\NormalTok{(}\AttributeTok{palette =} \StringTok{"Dark2"}\NormalTok{) }\SpecialCharTok{+}
  \FunctionTok{scale\_color\_brewer}\NormalTok{(}\AttributeTok{palette =} \StringTok{"Dark2"}\NormalTok{) }\SpecialCharTok{+}
  \FunctionTok{scale\_shape\_manual}\NormalTok{(}\AttributeTok{values =} \FunctionTok{c}\NormalTok{(}\DecValTok{21}\NormalTok{, }\DecValTok{22}\NormalTok{, }\DecValTok{24}\NormalTok{)) }\SpecialCharTok{+}
  \FunctionTok{labs}\NormalTok{(}\AttributeTok{x =} \StringTok{"Bill length (mm)"}\NormalTok{,}
       \AttributeTok{y =} \StringTok{"Bill depth (mm)"}\NormalTok{, }
       \AttributeTok{fill =} \StringTok{"Species"}\NormalTok{,}
       \AttributeTok{color =} \StringTok{"Species"}\NormalTok{,}
       \AttributeTok{shape =} \StringTok{"Species"}\NormalTok{) }\SpecialCharTok{+}
  \FunctionTok{theme\_minimal}\NormalTok{() }\SpecialCharTok{+}
  \FunctionTok{theme}\NormalTok{(}\AttributeTok{axis.title.x =} \FunctionTok{element\_text}\NormalTok{(}\AttributeTok{color =} \StringTok{"red"}\NormalTok{,}
                                    \AttributeTok{size =} \DecValTok{16}\NormalTok{)) }
\end{Highlighting}
\end{Shaded}

\includegraphics{README_files/figure-pdf/unnamed-chunk-22-1.pdf}

\begin{center}\rule{0.5\linewidth}{0.5pt}\end{center}

The customization of themes contains far too much content to walk
through here - hopefully this provides a springboard into the
possibilities! Always remember to revisit help pages and the cheatsheets
that have been provided along the way. Here are a couple annotated plots
that include a smattering of some other things that we didn't cover:

\begin{Shaded}
\begin{Highlighting}[]
\CommentTok{\# Graph bill length vs depth}
\FunctionTok{ggplot}\NormalTok{(}\AttributeTok{data =}\NormalTok{ penguins, }\FunctionTok{aes}\NormalTok{(}\AttributeTok{x =}\NormalTok{ bill\_length\_mm, }\AttributeTok{y =}\NormalTok{ bill\_depth\_mm)) }\SpecialCharTok{+}
  \CommentTok{\# Add points separated by species}
  \FunctionTok{geom\_point}\NormalTok{(}\FunctionTok{aes}\NormalTok{(}\AttributeTok{fill =}\NormalTok{ species,}
                 \AttributeTok{shape =}\NormalTok{ species),}
             \CommentTok{\# Change the size of the points}
             \AttributeTok{size =} \FloatTok{2.5}\NormalTok{) }\SpecialCharTok{+}
  \CommentTok{\# Add a lm line for each species}
  \FunctionTok{geom\_smooth}\NormalTok{(}\FunctionTok{aes}\NormalTok{(}\AttributeTok{color =}\NormalTok{ species),}
              \AttributeTok{method =} \StringTok{"lm"}\NormalTok{) }\SpecialCharTok{+}
  \CommentTok{\# Customize the fill and color of the geoms}
  \FunctionTok{scale\_fill\_brewer}\NormalTok{(}\AttributeTok{palette =} \StringTok{"Dark2"}\NormalTok{) }\SpecialCharTok{+}
  \FunctionTok{scale\_color\_brewer}\NormalTok{(}\AttributeTok{palette =} \StringTok{"Dark2"}\NormalTok{) }\SpecialCharTok{+}
  \CommentTok{\# Manually select the shape of the points}
  \FunctionTok{scale\_shape\_manual}\NormalTok{(}\AttributeTok{values =} \FunctionTok{c}\NormalTok{(}\DecValTok{21}\NormalTok{, }\DecValTok{22}\NormalTok{, }\DecValTok{24}\NormalTok{)) }\SpecialCharTok{+}
  \CommentTok{\# Make nicer axis and legend titles based on the aes() specified above}
  \FunctionTok{labs}\NormalTok{(}\AttributeTok{x =} \StringTok{"Bill length (mm)"}\NormalTok{,}
       \AttributeTok{y =} \StringTok{"Bill depth (mm)"}\NormalTok{, }
       \AttributeTok{fill =} \StringTok{"Species"}\NormalTok{,}
       \AttributeTok{color =} \StringTok{"Species"}\NormalTok{,}
       \AttributeTok{shape =} \StringTok{"Species"}\NormalTok{) }\SpecialCharTok{+}
  \CommentTok{\# Add in a plot title}
  \FunctionTok{ggtitle}\NormalTok{(}\StringTok{"Bill depth vs length of three penguin species"}\NormalTok{) }\SpecialCharTok{+}
  \CommentTok{\# Change the theme to theme\_minimal}
  \FunctionTok{theme\_minimal}\NormalTok{() }\SpecialCharTok{+}
  \FunctionTok{theme}\NormalTok{(}
    \CommentTok{\# Change the size of axis titles and text (both x and y at once!)}
    \AttributeTok{axis.title =} \FunctionTok{element\_text}\NormalTok{(}\AttributeTok{size =} \DecValTok{14}\NormalTok{),}
    \AttributeTok{axis.text =} \FunctionTok{element\_text}\NormalTok{(}\AttributeTok{size =} \DecValTok{12}\NormalTok{),}
    \CommentTok{\# Change the size of legend title and text}
    \AttributeTok{legend.text =} \FunctionTok{element\_text}\NormalTok{(}\AttributeTok{size =} \DecValTok{12}\NormalTok{),}
    \AttributeTok{legend.title =} \FunctionTok{element\_text}\NormalTok{(}\AttributeTok{size =} \DecValTok{12}\NormalTok{),}
    \CommentTok{\# Place the legend inside the plot, instead of the default outside}
    \AttributeTok{legend.position =} \StringTok{"inside"}\NormalTok{,}
    \CommentTok{\# Specify exactly where in x/y space (from 0 to 1) the legend should sit}
    \AttributeTok{legend.position.inside =} \FunctionTok{c}\NormalTok{(}\FloatTok{0.9}\NormalTok{, }\FloatTok{0.15}\NormalTok{),}
    \CommentTok{\# Color (not fill!) the legend {-} color refers to the outline here}
    \AttributeTok{legend.background =} \FunctionTok{element\_rect}\NormalTok{(}\AttributeTok{color =} \StringTok{"black"}\NormalTok{)}
\NormalTok{  )}
\end{Highlighting}
\end{Shaded}

\includegraphics{README_files/figure-pdf/unnamed-chunk-23-1.pdf}

\begin{Shaded}
\begin{Highlighting}[]
\CommentTok{\# Graph bill length vs depth}
\FunctionTok{ggplot}\NormalTok{(}\AttributeTok{data =}\NormalTok{ penguins, }\FunctionTok{aes}\NormalTok{(}\AttributeTok{x =}\NormalTok{ species, }\AttributeTok{y =}\NormalTok{ body\_mass\_g)) }\SpecialCharTok{+}
  \CommentTok{\# Put violin plots, which show the distribution of points like a histogram/density plot, underneath each set of points}
  \FunctionTok{geom\_violin}\NormalTok{(}\FunctionTok{aes}\NormalTok{(}\AttributeTok{fill =}\NormalTok{ species),}
              \CommentTok{\# make the violin plots quite transparent}
              \AttributeTok{alpha =} \FloatTok{0.25}
\NormalTok{              ) }\SpecialCharTok{+}
  \CommentTok{\# Add points separated by species}
  \FunctionTok{geom\_jitter}\NormalTok{(}\FunctionTok{aes}\NormalTok{(}\AttributeTok{fill =}\NormalTok{ species,}
                 \AttributeTok{shape =}\NormalTok{ species),}
             \CommentTok{\# Change the size of the points}
             \AttributeTok{size =} \FloatTok{2.5}\NormalTok{,}
             \CommentTok{\# Change the default amount that the points are "jittered" in the x (width) and y (height direction)}
             \AttributeTok{width =}\NormalTok{ .}\DecValTok{1}\NormalTok{,}
             \AttributeTok{height =} \DecValTok{0}\NormalTok{,}
             \CommentTok{\# Make the points slightly transparent}
             \AttributeTok{alpha =} \FloatTok{0.75}
\NormalTok{             ) }\SpecialCharTok{+}
  \CommentTok{\# Customize the fill and color of the geoms}
  \FunctionTok{scale\_fill\_brewer}\NormalTok{(}\AttributeTok{palette =} \StringTok{"Dark2"}\NormalTok{) }\SpecialCharTok{+}
  \FunctionTok{scale\_color\_brewer}\NormalTok{(}\AttributeTok{palette =} \StringTok{"Dark2"}\NormalTok{) }\SpecialCharTok{+}
  \CommentTok{\# Manually select the shape of the points}
  \FunctionTok{scale\_shape\_manual}\NormalTok{(}\AttributeTok{values =} \FunctionTok{c}\NormalTok{(}\DecValTok{21}\NormalTok{, }\DecValTok{22}\NormalTok{, }\DecValTok{24}\NormalTok{)) }\SpecialCharTok{+}
  \CommentTok{\# Make nicer axis and legend titles based on the aes() specified above}
  \FunctionTok{labs}\NormalTok{(}\AttributeTok{x =} \StringTok{"Species"}\NormalTok{,}
       \AttributeTok{y =} \StringTok{"Body mass (g)"}\NormalTok{, }
       \AttributeTok{fill =} \StringTok{"Species"}\NormalTok{,}
       \AttributeTok{color =} \StringTok{"Species"}\NormalTok{,}
       \AttributeTok{shape =} \StringTok{"Species"}\NormalTok{) }\SpecialCharTok{+}
  \CommentTok{\# Change the theme to theme\_minimal}
  \FunctionTok{theme\_minimal}\NormalTok{() }\SpecialCharTok{+}
  \FunctionTok{theme}\NormalTok{(}
    \CommentTok{\# Change the size of axis titles and text (both x and y at once!)}
    \AttributeTok{axis.title =} \FunctionTok{element\_text}\NormalTok{(}\AttributeTok{size =} \DecValTok{14}\NormalTok{),}
    \CommentTok{\# Add a little bit of space on the top (t) and the right (r) of the x and y axis titles respectively}
    \CommentTok{\# This separates them from the axis text a bit}
    \AttributeTok{axis.title.x =} \FunctionTok{element\_text}\NormalTok{(}\AttributeTok{margin =} \FunctionTok{margin}\NormalTok{(}\AttributeTok{t =} \DecValTok{10}\NormalTok{)),}
    \AttributeTok{axis.title.y =} \FunctionTok{element\_text}\NormalTok{(}\AttributeTok{margin =} \FunctionTok{margin}\NormalTok{(}\AttributeTok{r =} \DecValTok{10}\NormalTok{)),}
    \AttributeTok{axis.text =} \FunctionTok{element\_text}\NormalTok{(}\AttributeTok{size =} \DecValTok{12}\NormalTok{),}
    \CommentTok{\# Change the size of legend title and text}
    \AttributeTok{legend.text =} \FunctionTok{element\_text}\NormalTok{(}\AttributeTok{size =} \DecValTok{12}\NormalTok{),}
    \AttributeTok{legend.title =} \FunctionTok{element\_text}\NormalTok{(}\AttributeTok{size =} \DecValTok{12}\NormalTok{)}
\NormalTok{  )}
\end{Highlighting}
\end{Shaded}

\includegraphics{README_files/figure-pdf/unnamed-chunk-24-1.pdf}

\section{2) Make your abalone figure ready for prime
time!}\label{make-your-abalone-figure-ready-for-prime-time}

\begin{figure}[H]

{\centering \includegraphics{blackabalone.jpg}

}

\caption{Black abalone}

\end{figure}%

Next, it's time to practice what you've learned!

Please take a look at the feedback you got on the abalone figure (go
back to the
\href{https://docs.google.com/presentation/d/1wAGmbArAW2YKmufrBbzCQ3BcDdwDknzgoRHYEmFXJko/edit?usp=sharing}{Google
slides}, which now have the comments pasted in the speaker notes).
Decide which aspects of the feedback you find useful, combined with your
own thoughts on how to make the figure more effective as an explanatory
graph.

\subsubsection{Q2.1: Pick an audience for your explanatory
graph.}\label{q2.1-pick-an-audience-for-your-explanatory-graph.}

This could be everything from another scientist to a policymaker, a
community member, your grandmother, or anyone else. Please write a short
explanation of who this person or group of people is.

A: we decided our audience is your grandmother

\begin{center}\rule{0.5\linewidth}{0.5pt}\end{center}

\subsubsection{Q2.2: Remake your original graph from
Monday.}\label{q2.2-remake-your-original-graph-from-monday.}

This should be as simple as copy and pasting the code you used on Monday
(though remember to read in the data!). You'll need to create a new code
block below here (remember how to do that?).

\begin{Shaded}
\begin{Highlighting}[]
\NormalTok{ab\_data }\OtherTok{=} \FunctionTok{read.csv}\NormalTok{(}\StringTok{"abalone\_landings.csv"}\NormalTok{)}
\NormalTok{ab\_plot }\OtherTok{=} \FunctionTok{ggplot}\NormalTok{(}\AttributeTok{data =}\NormalTok{ ab\_data, }\AttributeTok{mapping =} \FunctionTok{aes}\NormalTok{(}\AttributeTok{x =}\NormalTok{ Year, }\AttributeTok{y =}\NormalTok{ Abalone\_Landings\_lbs)) }\SpecialCharTok{+} 
  \FunctionTok{geom\_point}\NormalTok{(}\FunctionTok{aes}\NormalTok{(}\AttributeTok{color =}\NormalTok{ Abalone\_Species)) }\SpecialCharTok{+}
  \FunctionTok{facet\_wrap}\NormalTok{(Abalone\_Species }\SpecialCharTok{\textasciitilde{}}\NormalTok{., }\AttributeTok{scales =} \StringTok{"free\_y"}\NormalTok{) }\SpecialCharTok{+}
  \FunctionTok{geom\_smooth}\NormalTok{(}\FunctionTok{aes}\NormalTok{(}\AttributeTok{color =}\NormalTok{ Abalone\_Species)) }

\NormalTok{ab\_plot}
\end{Highlighting}
\end{Shaded}

\includegraphics{README_files/figure-pdf/unnamed-chunk-25-1.pdf}

\begin{center}\rule{0.5\linewidth}{0.5pt}\end{center}

\subsubsection{Q2.3: Describe how you would like to modify the figure to
make it more effective as an explanatory graph for the audience you
chose in
Q2.1.}\label{q2.3-describe-how-you-would-like-to-modify-the-figure-to-make-it-more-effective-as-an-explanatory-graph-for-the-audience-you-chose-in-q2.1.}

The past two activities have provided a brief sampling of the
possibilities of customizing graphs using ggplot. You are welcome to
explore other possibilities, too, either on your own or by talking with
the teaching team.

\begin{center}\rule{0.5\linewidth}{0.5pt}\end{center}

\subsubsection{Q2.4 Make the explanatory
graph!}\label{q2.4-make-the-explanatory-graph}

Use what you have learned and make a Really Nice Graph for your
audience! Go wild with making it pretty.

\begin{Shaded}
\begin{Highlighting}[]
\NormalTok{ab\_data }\OtherTok{=} \FunctionTok{read.csv}\NormalTok{(}\StringTok{"abalone\_landings.csv"}\NormalTok{)}
\NormalTok{ab\_plot }\OtherTok{=} \FunctionTok{ggplot}\NormalTok{(}\AttributeTok{data =}\NormalTok{ ab\_data, }\AttributeTok{mapping =} \FunctionTok{aes}\NormalTok{(}\AttributeTok{x =}\NormalTok{ Year, }\AttributeTok{y =}\NormalTok{ Abalone\_Landings\_lbs)) }\SpecialCharTok{+} 
  \FunctionTok{geom\_point}\NormalTok{(}\FunctionTok{aes}\NormalTok{(}\AttributeTok{color =}\NormalTok{ Abalone\_Species)) }\SpecialCharTok{+}
  \FunctionTok{geom\_smooth}\NormalTok{(}\FunctionTok{aes}\NormalTok{(}\AttributeTok{color =}\NormalTok{ Abalone\_Species)) }\SpecialCharTok{+} 
  \FunctionTok{scale\_color\_manual}\NormalTok{(}\AttributeTok{values =} \FunctionTok{c}\NormalTok{(}\StringTok{"black"}\NormalTok{,}\StringTok{"green"}\NormalTok{,}\StringTok{"pink"}\NormalTok{, }\StringTok{"red"}\NormalTok{,}\StringTok{"gray"}\NormalTok{)) }\SpecialCharTok{+}
   \FunctionTok{labs}\NormalTok{(}\AttributeTok{x =} \StringTok{"Year"}\NormalTok{,}
       \AttributeTok{y =} \StringTok{"Abalone Landings (lbs)"}\NormalTok{, }
       \AttributeTok{fill =} \StringTok{"Species"}\NormalTok{,}
       \AttributeTok{color =} \StringTok{"Species"}\NormalTok{,}
       \AttributeTok{shape =} \StringTok{"Species"}\NormalTok{,}
       \AttributeTok{title =} \StringTok{"Abalone Landings Through The Years"}\NormalTok{) }\SpecialCharTok{+}

  \FunctionTok{theme\_minimal}\NormalTok{() }
  

\NormalTok{ab\_plot}
\end{Highlighting}
\end{Shaded}

\includegraphics{README_files/figure-pdf/unnamed-chunk-26-1.pdf}

\subsubsection{Q2.5: Explain your graph}\label{q2.5-explain-your-graph}

Write a couple sentences explaining the message you hope your audience
will take away from seeing your graph.

A: this graph is much more appealing for your grandmother to understand,
less individual facets to look at and colors that match that of the
ablone species. All wrapped up with nice text formatting and the minimal
theme.

\section{Wrap up and submit}\label{wrap-up-and-submit}

As in the previous lesson, render this Quarto document to a PDF (if you
can; see guidance at the top) and submit this through GradeScope.




\end{document}
